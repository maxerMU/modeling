\chapter{Практическая часть}

На листинге \ref{lst:criterium} представлен код для моделирования работы системы.

\begin{lstinputlisting}[label=lst:criterium,caption=Критерий для оценки случайности, language=c, firstline=1, lastline=35]{../mycriteirium.cpp}
\end{lstinputlisting}

На листинге \ref{lst:alg} представлен код алгоритмического способа.
\begin{lstinputlisting}[label=lst:alg,caption=Алгоритмический способ, language=c, firstline=6, lastline=33]{../linearcongruentrandomizer.cpp}
\end{lstinputlisting}

На листинге \ref{lst:table} представлен код табличного способа.
\begin{lstinputlisting}[label=lst:table,caption=Табличный способ, language=c, firstline=10, lastline=36]{../tablerandomizer.cpp}
\end{lstinputlisting}

Результат работы алгоритмического и табличного способов и критерий для их проверки представлены на рисунке \ref{img:res1}. 

\imgs{res1}{H}{0.4}{Табличный и алгоритмический способы}

Результат для последовательности от 0 до 9 представлен на рисунке \ref{img:res2}. 
\imgs{res2}{H}{0.4}{Последовательность от 0 до 9}

Результат для последовательности, состоящей из одних единиц, представлен на рисунке \ref{img:res3}. 
\imgs{res3}{H}{0.4}{Последовательность из одних единиц}

Результат для последовательности чисел 1 и 9 представлен на рисунке \ref{img:res4}. 
\imgs{res4}{H}{0.4}{Последовательность чисел 1 и 9}