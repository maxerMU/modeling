\chapter{Теоретический раздел}

Существует три основных способа получения последовательностей случайных чисел:
\begin{enumerate}
	\item аппаратный;
	
	\item табличный (файловый);
	
	\item алгоритмический.
\end{enumerate}

В рамках лабораторной работы рассмотрены алгоритмический и табличный способы. 

\section{Алгоритмический способ}
Этот подход основан на базе специальных алгоритмов. К ним относятся:
\begin{itemize}
	\item метод серединных произведений;
	
	\item метод перемешивания;
	
	\item линейный конгруэнтный метод.
\end{itemize}

Было принято решение взять последний для генерации последовательности псевдослучайных чисел.

В этом методе каждое следующее число рассчитывается на основе предыдущего по формуле (\ref{formula1}).
\begin{equation}\label{formula1}
R_{n + 1} = (a \cdot R_n + b)\;mod\;N,\, n \geq 1
\end{equation}
где a, b -- коэффициенты, N -- модуль.

Для качественного генератора требуется подобрать подходящие коэффициенты. Например, в таблице \ref{k} приведены некоторые из них.
\begin{table}[h]
	\begin{center}
		\caption{Примеры коэффициентов}
		\label{k}
		\begin{tabular}{| p{3cm} | p{3cm} | p{3cm}|}
			\hline
			\textbf{a} 			& \textbf{b} 	& \textbf{N} \\
			\hline
			106 				& 1283 			& 6075 \\ 
			\hline
			430 				& 2531  		& 11979 \\ 
			\hline
			84589 				& 15989 		& 217728 \\ 
			\hline
			1103515245 			& 12345 		& $2^{31}$ \\ 
			\hline
			... 				& ... 			& ... \\ 
			\hline
		\end{tabular}
	\end{center}
\end{table} 

\section{Табличный способ}
В качестве источника случайных чисел используют специально заранее составленные таблицы, содержащие проверенные данные.

\section{Критерий оценки}
В рамках лабораторной работы был выбран собственный критерий, который вычисляется следующим образом:
\begin{enumerate}
	\item массив с размером равным длине интервала выборки случайного значения заполняется значениями, равными числу выпадений каждого числа интервала;
	\item у полученной последовательности считаются выборочное среднее и выборочная дисперсия;
	\item считается дисперсия последовательности, полученной в случае, когда все значения, полученные с помощью алгоритма генерации псевдослучайных чисел, были одинаковые;
	\item результат получается по формуле $ res = 1 - \frac{freq\_disp}{max\_disp}, $ где числитель дроби получен на шаге 2, а знаменатель -- на шаге 3.
\end{enumerate}

Чем ближе полученный результат к 1, тем более случайной является входная последовательность.