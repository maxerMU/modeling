\chapter{Практическая часть}

На листинге \ref{lst:markov} представлен код для моделирования работы системы.

\begin{lstinputlisting}[label=lst:markov,caption=Построение графиков для равномерного распределения, language=python, firstline=1, lastline=39]{../src/markov.py}
\end{lstinputlisting}

На листинге \ref{lst:ui} представлен код для создания интерфейса.

\begin{lstinputlisting}[label=lst:ui,caption=Построение графиков для равномерного распределения, language=python, firstline=7, lastline=55]{../src/main.py}
\end{lstinputlisting}

Результат работы программы для системы из 3 состояний представлен на рисунке \ref{img:res3}. 

\imgs{res3}{H}{0.5}{Система из трех состояний}

Здесь первый столбец отвечает за номер состояния, второй за время через, которое вероятность нахождения системы в этом состоянии стала стабильной, третий -- эта вероятность.

Результат работы программы для системы из 5 состояний представлен на рисунке \ref{img:res5}.

\imgs{res5}{H}{0.5}{Система из пяти состояний}