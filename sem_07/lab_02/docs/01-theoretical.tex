\chapter{Теоретический раздел}

Случайный процесс, протекающий в некоторой системе S, называется \textbf{марковским}, если он обладает свойством: для каждого момента времени $t_0$ вероятность любого состояния системы в будущем (при $t > t_0$) зависит только от её состояния в настоящем, и не зависит от того, когда и каким образом система пришла в это состояние, т. е. не зависит от того, как процесс развивался в прошлом. 

Для Марковских процессов обычно составляют уравнения Колмогорова. 

Общий вид:
$$
F = (p'(t), p(t), \Lambda) = 0,
$$
где $\Lambda = \lambda_1, \lambda_2, ..., \lambda_n$ - набор коэффициентов.

Вероятностью i-го состояния называется вероятность $p_i(t)$ того, что в момент t система будет находиться в состоянии $S_i$. К системе может быть добавлено условие нормировки: для любого
момента t сумма вероятностей всех состояний равна единице:
$$
\sum_{i = 1}^n p_i = 1.
$$

Для того, чтобы решить поставленную задачу, необходимо составить систему
уравнений Колмогорова. Все уравнения составляются по определённым правилам.
\begin{itemize}
	\item В левой части каждого уравнения стоит производная вероятности i-ого состояния.
	
	\item В правой части содержится столько членов, сколько переходов, связанных с данным состоянием. Если переход из состояния, то соответствующий член имеет знак минус, а, если наоборот, то плюс. 
	
	\item Каждый член равен произведению плотности вероятности перехода (т.е. интенсивности), соответствующей рассматриваемой стрелки, на вероятность того состояния, из которого исходит стрелка.
\end{itemize}