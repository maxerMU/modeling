\chapter{Теоретический раздел}

\section{Равномерное распределение}

Равномерное распределение -- распределение случайной величины, принимающей значения, принадлежащие некоторому промежутку конечной длины, характеризующееся тем, что плотность вероятности на этом промежутке всюду постоянна. Функция равномерного распределения представлена формулой \ref{uniform:F}.

\begin{equation}\label{uniform:F}
F(x) = \begin{Bmatrix}
$$
	0, & x\leqslant a \\
	\frac{x - a}{b - a}, & a\leq x\leq b \\
	1, & x\geq b
$$
\end{Bmatrix}
\end{equation}

Функция плотности равномерного распределения представлена формулой \ref{uniform:f}.

\begin{equation}\label{uniform:f}
f(x) = \left\{\begin{matrix}
$$
	\frac{1}{b - a}, & a\leq x\leq b \\
	0, & (x < a) or (x > b)
$$
\end{matrix}\right.
\end{equation}

\section{Экспоненциальное распределение}

Экспоненциальное распределение является частным случаем гамма распределения с параметрами $a = 1$ и  $b = \frac{1}{\lambda}$.
Функция экспоненциального распределения представлена формулой \ref{exp:F}.

\begin{equation}\label{exp:F}
F(x) = \left\{\begin{matrix}
$$
1 - e^{-\lambda x}, & x \geq 0 \\
0, & x < 0
$$
\end{matrix}\right.
\end{equation} 

Функция плотности экспоненциального распределения представлена формулой \ref{exp:f}.
\begin{equation}\label{exp:f}
F(x) = \left\{\begin{matrix}
$$
\lambda e^{-\lambda x}, & x \geq 0 \\
0, & x < 0
$$
\end{matrix}\right.
\end{equation}

\section{Пошаговый принцип ($\Delta t$)}
Этот принцип заключается в последовательном анализе состояний всех блоков системы в момент $t + \Delta t$. При этом новое состояние блоков определяется в соответствии с их алгоритмическим описанием. 

Недостаток: значительные временные затраты на реализацию моделирования системы. А также при недостаточно малом $\Delta t$ отдельные события в системе могут быть пропущены, что может повлиять на адекватность результатов.

\section{Событийный принцип}
Состояние отдельных устройств изменяются в дискретные моменты времени, совпадающие с моментами времени поступления сообщений в систему, временем окончания обработки задачи и т.д.

При использовании событийного принципа состояние всех блоков системы анализируется лишь в момент проявления какого-либо события. Моменты наступления следующего события определяются минимальным значением из списка событий.
