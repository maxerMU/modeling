\chapter{Практическая часть}

На листинге \ref{lst:uniform} представлен код генератора событий.

\begin{lstinputlisting}[label=lst:uniform,caption=Генератор событий, language=python, firstline=23, lastline=43]{../src/modeller.py}
\end{lstinputlisting}

На листинге \ref{lst:oa} представлен код обслуживающего аппарата.

\begin{lstinputlisting}[label=lst:oa,caption=Обслуживающий аппарат, language=python, firstline=45, lastline=86]{../src/modeller.py}
\end{lstinputlisting}

На листинге \ref{lst:gen} представлен код генераторов равномерного и экспоненциального распределений.

\begin{lstinputlisting}[label=lst:gen,caption=Генераторы, language=python, firstline=4, lastline=20]{../src/modeller.py}
\end{lstinputlisting}

На листинге \ref{lst:step} представлен код пошагового принципа.

\begin{lstinputlisting}[label=lst:step,caption=Пошаговый принцип, language=python, firstline=115, lastline=135]{../src/modeller.py}
\end{lstinputlisting}

На листинге \ref{lst:event} представлен код событийного принципа.

\begin{lstinputlisting}[label=lst:event,caption=Событийный принцип, language=python, firstline=95, lastline=113]{../src/modeller.py}
\end{lstinputlisting}

Пример работы системы с использованием пошагового принципа представлен на рисунке \ref{img:time}. 

\imgs{time}{H}{0.5}{Пошаговый принцип}

Пример работы системы с использованием событийного принципа представлен на рисунке \ref{img:event}. 

\imgs{event}{H}{0.5}{Событийный принцип}

